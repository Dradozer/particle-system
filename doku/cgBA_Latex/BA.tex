\documentclass[intern,palatino]{cgBA}

\author{Sebastian Gaida}
\title{Simulation von Rauch mittels Partikelsystem}
\zweitgutachter{Bastian Krayer MSc. }
\zweitgutachterInfo{(Institut für Computervisualistik, AG Computervisualistik)}
\externLogo{7.46cm}{logos/UniLogoNeu}
\externName{DIN: NewTechnologies}

\usepackage{acronym}
\usepackage{hyperref}
\usepackage{url}

\begin{document}

	\maketitle
	\newpage
	\pagenumbering{roman}
	\tableofcontents
	\clearpage         % oder \cleardoublepage bei zweiseitigem Druck
	% \listoffigures   % fuer ein eventuelles Abbildungsverzeichnis
	\pagenumbering{arabic}
	\bibliographystyle{alphadin}

%-------------------------------------------------------------------------------

\section*{Abstract}

In dieser Arbeit wird auf die realistische Simulation von Rauch eingegangen. Dabei bezieht sich die Arbeit hauptsächlich auf die Simulationen von Müller et al.\cite{muller2003particle} und Ren et al.\cite{ren2016fast}. Die Simulation wurde mittels C++, der \ac{abk:OGL} und Compute Shadern erstellt. Hierbei wurde das \ac{abk:SPH} Verfahren genutzt und die Möglichkeiten zur Beschleunigung auf der \ac{abk:GPU} untersucht.
\newline \newline
This paper deals with the realistic simulation of smoke. The work refers mainly to the simulations of Müller et al.\cite{muller2003particle} and Ren et al.\cite{ren2016fast}. The simulation was created using C++, the \ac{abk:OGL} and compute shaders. Here the \ac{abk:SPH} method was used and the possibilities to accelerate it on the \ac{abk:GPU} were investigated.
\newpage

%-------------------------------------------------------------------------------

\section{Vorwort}

Vor dem Beginn der vorliegenden Bachelorarbeit möchte ich mich zunächst bei einigen Personen bedanken die mich während der Arbeit unterstützt haben.\newline \newline
Zunächst einmal bedanke ich mich bei Prof. Dr.-Ing. Stefan Müller und Bastian Krayer MSc. für die großartige Betreuung meiner Arbeit.\newline
Außerdem möchte ich mich bei Pascal Bendler bedanken, der mich tatkräftig beim debugging unterstützt hat.\newline
Das größte Dankeschön geht aber an den Freund, der mich immer wieder dazu motiviert hat weiter zu arbeiten und nach alternativen Möglichkeiten zu suchen.
\newpage

%-------------------------------------------------------------------------------

\section{Einleitung}

%-------------------------------------------------------------------------------

\section{State of the Art}
\subsection{Vektorfelder}
\subsection{Partikelsystem}

%-------------------------------------------------------------------------------

\section{Fluidsimulation}
\subsection{Dichte}
\subsection{Viskosität}
\subsection{Druck}
\subsection{Auftrieb}

%-------------------------------------------------------------------------------

\section{Beschleunigung}
\subsection{Grid-basiertes-Verfahren}
\subsection{Sortierverfahren}
\subsection{Vergleich}

%-------------------------------------------------------------------------------

\section{Ergebnis}

%-------------------------------------------------------------------------------

\section{Fazit}

%-------------------------------------------------------------------------------

\begin{acronym}
	\acro{abk:OGL}[OpenGL]{Open Graphics Libary}
	\acro{abk:SPH}[SPH]{Smoothed Particle Hydrodynamics}
	\acro{abk:GPU}[GPU]{Graphics Processing Unit}
\end{acronym}

\bibliography{lib}

\end{document}